\chapter{Objetivos del trabajo}

El principal objetivo de este TFG es el conocimiento de una plataforma de carácter didáctico orientada a FPGAs de la familia Zynq de Xilinx para la 
realización de diferentes ejercicios prácticos útiles en el aprendizaje en asignaturas relacionadas con el diseño de sistemas 
basados en dispositivos hardware reconfigurables.

La empresa Xilinx es una empresa líder en el desarrollo de herramientas no sólo a nivel de síntesis RT-lógica, sino 
también de síntesis de alto nivel y particularmente para co-diseño HW/SW de sistemas empotrados en FPGAs basado en aplicaciones y 
descripciones tipo C/C++. Sus herramimentas se han utilizado en asignaturas de perfil más avanzado y resulta conveniente la 
migración a este tipo de plataformas de las prácticas que se realizan en otras asignaturas de perfil más básico, con intención de 
favorecer al estudio con una misma plataforma basada en la herramienta Vivado y una tarjeta de desarrollo basada en
FPGAs Zynq.
 
Se toma como referencia la asignatura "Desarrollo de hardware digital" de la especialidad de Ingeniería de Computadores que pertenece al 
grado de Informática, cuyos contenidos se corresponden fundamentalmente con el aprendizaje de la metodología de diseño de sistemas basados en FPGAs con herramientas de síntesis 
automática y verificación a partir de descripciones VHDL en el nivel RT, para el análisis y diseño de módulos hardware específicos, tales 
como procesadores específicos, memorias, y módulos de interfaz y comunicaciones.

Y por último se establecen siguientes objetivos:
\begin{enumerate}
    \item Estudio y descripción de cómo se realiza en Vivado la metodología propia de flujo de diseño con FPGAs a partir de descripciones RT en VHDL.
    \item Descripción módulos de especial interés en la plataforma para la realización de prácticas (procesador, memoria, interfaz VGA, generación de reloj).
    \item Realización de dos casos prácticos planteados en la asignatura para comprobar la utilidad y el correcto funcionamiento de los módulos que en este 
    momento integran la plataforma junto con la tarjeta Zybo y la herramienta Vivado.
\end{enumerate}