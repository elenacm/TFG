\chapter{Conclusiones y vías futuras}

Como resultado, se han conseguido los objetivos marcados, se ha conocido y estudiado una plataforma de carácter didáctico, Vivado, a partir de la documentación facilitada 
por el fabricante Xilinx, para la realización de ejercicos prácticos que sean útiles para el aprendizaje del diseño de sistemas basados en FPGAs a partir de descripciones VHDL para síntesis RT. 

Xilinx es la empresa líder en la fabricación de FPGAs además de en herramientas de desarrollo en los distintos niveles de síntesis automática. Además se 
ha analizado la plataforma Vivado junto con la tarjeta Zybo para el estudio de la metodología y flujo de diseño de sistemas basados en FPGAs a partir de descripciones VHDL para síntesis RT.

Se ha realizado un ejemplo sencillo (contador de 4 bits) con el que se ha validado el flujo de diseño y después se han realizado un par de ejemplos prácticos, un computador 
básico y la visualización de una bola en la pantalla, validando módulos VHDL útiles en las prácticas de laboratorio de la asignatura ``Desarrollo de Hardware Digital'', como un porcesador y 
un controlador VGA. Además para la realización de estos ejemplos se ha tenido que generar módulos de memoria y de generación de reloj a partir del catálogo de componentes 
IP de Vivado, los cuales se han validado y se ha comprobado su correcto funcionamiento.

Es cierto que al principio hubo algunos problemas con respecto a conocer la tarjeta usada y la plataforma de desarrollo, además de con la realización 
de los ejemplos prácticos. Todo esto ha tenido un buen resultado, por lo que se puede decir que se han cumplido los objetivos de este trabajo de fin de grado, resultando viable 
y satisfactoria la migración de las prácticas de la asignatura DHD para su realzación con la herramienta Vivado y con la tarjeta de desarrollo Zybo.

Como posibles vías futuras se puede considerar la ampliación del número de módulos VHDL para integración de sistemas de interés pedagógico como módulos de interfaz PS/2 para teclado y 
ratón o módulos sencillos de comunicaciones como I2C o UART. Otra podría ser ampliar la plataforma con módulos descritos en C/C++ útiles en asignaturas de perfil más 
avanzado que emplean Vivado HLS. Incluso se podría utilizar otra tarjeta de caracter didáctico como la Zybo Z7 que es la versión actualizada de la Zybo.

La herramienta Vivado se emplea en algunas asignaturas del Máster en el que me he matriculado. Este Máster proporciona una visión general del estado del arte de las 
actuales y futuras tecnologías electrónicas, así como una base específica y metodológica para poder realizar labores de investigación y desarrollo en el área de los 
sistemas electrónicos.

Algunas de las asignaturas de este máster contienen entre otros temas la arquitectura interna de las FPGAs así como las tecnologías de fabricación más modernas y los 
principales fabricantes de estas. También se usa la herramienta Vivado usada en este trabajo para realizar prácticas con una placa de Digilent que incluye una 
FPGA Artix-7 de Xilinx. 