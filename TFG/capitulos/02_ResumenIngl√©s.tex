In this final degree proyect, a study of a educational platform for the development of systems based on Xilinx FPGAs has been carried out. 
an FPGA is a semiconductor device based on configurable logic blocks that are connected by programmable interconnections. To program 
them there are three main technologies, the Antifuse technology, that is not reprogrammable, the SRAM technology that is reprogrammable but 
with volatile memories and the Flash technology that is reprogrammable but with non-volatile memory.

Currently FPGAs can be used to implement systems in different areas such as Aerospace and Defense, Data Processing Center, Industry, 
Medicine or Communications.

Among the main FPGA manufacturers, Xilinx stands out as the main manufacturer, followedby Intel. The characteristics of its devices and 
development software tools are described in this report.

The board that will be use in this project for the development of this project is the ZYBO board that includes an FPGA of the Xilinx 
Zynq-7000 family. This FPGA includes a processing system based on embedded ARM cores, ``on-chip'' memory, external memory interfaces and 
generic logic based on SRAM technology. The architectureof this FPGA allowsthe implementation of custom logic to configure specific hardware 
modules and the execution of software in the embedded processors, which are explained in the memory development.

Xilinx has been a pioneer in the commercialization of high-level synthesis tools to describe hardware components from C / C ++ descriptions with 
the Vivado HLS module that has been used in advanced profile courses. In this work, the migration to Vivado of laboratory practices of a subject of 
a more basic nature is proposed where the design of components is studied from RT-logic synthesis tools in which hardware description languages such 
as VHDL or Verilog are used. The language used in this work is VHDL.

The objective of this work is to know the Vivado platform and carry out a design flow based on the board itself. Vivado is prepared for the synthesis 
and analysis of HDL designs. To do this, you must know the possibilities that Vivado has for each phase of the design flow. Thus, it has been 
studied how the different phases of the design flow are implemented with this tool. Next, a description of the main modules that will be used 
in the subsequent realization of practical cases has been made. These modules are the most relevant for the purpose of this project, there are many more. 
Specifically, they are a educational processor, a RAM memory, a clock generator module and a VGA controller.

Then, as practical cases, a basic computer is detailed, in which the RAM memoru block is used, and the on-screen display of a ball moving in it, 
in which the clock generator module and the VGA controller are used.

Finally, the final conclusion of the work is carried out, where it is shown if the objectives have been achieved and also other future routes 
that can be carried out from this project are included.