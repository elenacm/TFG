En este trabajo de fin de grado se ha realizado un estudio de una plataforma didáctica para desarrollo de sistemas basados en FPGAs de Xilinx. 
Una \textbf{FPGA} es un dispositivo semiconductore basado en matrices de bloques lógicos configurables que están conectados mediante interconexiones programables. 
Para programarlas existen tres principales tecnologías, la tecnología Antifusible que no es reprogramable, la tecnología SRAM que son reprogramables 
pero con memorias son volátiles y la tecnología Flash que es reprogramable pero con memoria no volátil. 

Actualmente las FPGAs pueden ser usadas para implementar sistemas en distintos ámbitos como por ejemplo Aeroespacial y defensa, 
Centro de procesamiento de datos, Industria, Medicina o Comunicaciones.

Dentro de los principales fabricantes de FPGAs, destaca \textbf{Xilinx} como el principal fabricante, seguido de Intel. Las características de sus dispositivos 
y herramientas sofwtare de desarrollo son descritas en esta memoria.

La tarjeta que se va a usar en este proyecto para el desarrollo de este proyecto es la tarjeta \textbf{ZYBO} que incluye una FPGA de la familia Zynq-7000 de Xilinx. Esta FPGA incluye 
un sistema de procesamiento basado en cores ARM empotrados, memoria ``on-chip'', interfaces de memoria externa y lógica genérica basada en tecnología SRAM. La arquitectura 
de esta FPGA permite la implementación de lógica personalizada para configurar módulos hardware específicos y la ejecución de software en los procesadores empotrados, los cuales 
son explicados en el desarrollo de la memoria. 

Xilinx ha sido una empresa pionera en la comercialización de herramientas de síntesis de alto nivel para describir componentes hardware a partir de descripciones C/C++ 
con el módulo Vivado HLS que se ha utilizando en asignaturas de perfil avanzado. En este trabajo se plantea la migración a Vivado de prácticas de laboratorio 
de un asignatura de carácter más básico donde se estudia el diseño de componentes a partir de herramientas de síntesis RT-lógica en las que se utilizan lenguajes 
de descripción hardware tipo \textbf{VHDL} o \textbf{Verilog}. El lenguaje usado en este trabajo es VHDL.

El objetivo de este trabajo es conocer la plataforma \textbf{Vivado} y realizar un flujo de diseño apoyándonos en la propia tarjeta. Vivado está preparado 
para la síntesis y análisis de diseños HDL. Para ello, hay que conocer las posibilidades que tiene Vivado para cada fase del flujo de diseño. Así, se ha estudiado cómo 
se implementan las diferentes fases del flujo de diseño con esta herramienta. A continuación, se ha realizado una descripción de los principales módulos que se van a 
usar en la posterior realización de casos prácticos. Estos módulos son los más relevantes para la finalidad de este proyecto, existiendo muchos más. En concreto son, 
un procesador didáctico, una memoria RAM, un módulo generador de reloj y un controlador VGA.

Después, como casos prácticos se detallan un computador básico, en el que se usa el bloque de memoria RAM, y la visualización en pantalla de una bola moviéndose 
en ella, en el que se usa el módulo generador de reloj y el controlador VGA.

Por último se realiza la conclusión final del trabajo, donde se muestra si se han conseguido los objetivos planteados y además se incluyen otras vías futuras que se 
pueden realizar a partir de este proyecto.