\chapter{Motivación e introducción}


\section{Sistemas basados en dispositivos FPGAs} 

Las FPGAs (\textit{Field Programmable Gate Arrays}), son dispositivos semiconductores basados en matrices de bloques lógicos configurables
(\textbf{CLB}) que están conectados mediante interconexiones programables \cite{fpga_xilinx}. Estas FPGAs pueden ser reprogramadas para 
algún trabajo específico o para cambiar los requisitos de funcionalidad después de su fabricación. Algunas pueden ser programadas 
una sola vez mientras que otras pueden ser reprogramadas una y otra vez. A estos dispositivos que son programados una única vez 
son referidos como \textbf{OTP} (\textit{one-time programmable}).

\textit{Field Programmable}, se refiere al hecho de que su programación se hace "\textit{en el campo}" a diferencia de otros dispositivos 
que su funcionalidad está programada por el fabricante \cite{maxfield1}.

Hay muchos tipos diferentes de circuitos integrados digitales, entre los que destacamos \textbf{PLDs} (\textit{Programmable Logic Devices}), 
\textbf{ASICs} (\textit{Application-Specific Integrated circuits}), \textbf{ASSPs} (\textit{Application-Specific Standard Parts}) y \textbf{FPGAs}.

Los \textbf{PLDs} son dispositivos con una arquitectura interna predeterminada por el fabricante, creados para ser configurados por 
ingenieros en el campo para realizar diferentes funciones. En comparación a las \textbf{FPGAs}, contiene un número limitado de puertas lógicas 
y las funciones que se suelen implementar son más pequeñas y simples.

Por otro lado los \textbf{ASICs} y los \textbf{ASSPs} contienen cientos de millones de puertas lógicas y se usan para crear grandes y complejas 
funciones. Ambos están basados en los mismo procesos de diseño y tecnologías. La única diferencia es que un \textbf{ASIC} está diseñado y fabricado 
para ser usado por una compañía específica, mientras que un \textbf{ASSP} es comercializado a muchos clientes.

Así, las \textbf{FPGAs} se encuentran entre los \textbf{PLDs} y los \textbf{ASICs} porque su funcionalidad puede ser diseñada en el campo como 
los \textbf{PLDs}, pero pueden contener millones de puertas lógicas y ser usadas para implementar funciones complejas que previamente sólo 
podían ser realizadas usando \textbf{ASICs}. 

El coste de un diseño de \textbf{FPGA} es mucho menor que el de uno de un \textbf{ASIC}. Al mismo tiempo, los cambios de diseño implementados 
son más fáciles en \textbf{FPGAs} y el tiempo de comercialización es más rápido \cite{maxfield2}.


\subsection{Aplicaciones}

A mediados del año 1980 llegaron las FPGAs, que eran usadas para implementar lógicas simples, máquinas de estados con una complejidad media 
y tareas de procesamiento de datos. A principios de los 90s, el mercado en el que se vendían se extendió al área de las telecomunicaciones 
debido a que el tamaño y sofisticación de las mismas empezaron a crecer. A finales de los 90s, el uso de las FPGAs en aplicaciones de consumo 
e industriales tuvo un enorme crecimiento.

Las FPGAs a menudo son utilizadas para crear prototipos de diseños ASIC o para tener un plataforma hardware donde verificar la implementación 
física de nuevos algoritmos \cite{maxfield1}. 

Actualmente se pueden encontrar FPGAs de alto rendimiento con millones de puertas. Algunos de estos dispositivos tienen núcleos de 
microprocesador integrados, dispositivos de entrada-salida de alta velocidad y similares. El resultado es que actualmente las FPGAs pueden ser 
usadas para implementar casi cualquier cosa en distintos ámbitos como por ejemplo:

\begin{itemize}
    \item \textbf{Aeroespacial y defensa} 
    \item \textbf{Audio} 
    \item \textbf{Automotriz} 
    \item \textbf{Broadcast} 
    \item \textbf{Electrónica} 
    \item \textbf{Centro de datos} 
    \item \textbf{Computación de alto rendimiento} 
    \item \textbf{Industria} 
    \item \textbf{Medicina} 
\end{itemize}

\section{Niveles de síntesis automática}  

\section{Plataformas de desarrollo} 

\section{Estructura de la memoria} 
 
 
 